\PassOptionsToPackage{unicode=true}{hyperref} % options for packages loaded elsewhere
\PassOptionsToPackage{hyphens}{url}
%
\documentclass[doc]{apa6}
\usepackage{lmodern}
\usepackage{amssymb,amsmath}
\usepackage{ifxetex,ifluatex}
\usepackage{fixltx2e} % provides \textsubscript
\ifnum 0\ifxetex 1\fi\ifluatex 1\fi=0 % if pdftex
  \usepackage[T1]{fontenc}
  \usepackage[utf8]{inputenc}
  \usepackage{textcomp} % provides euro and other symbols
\else % if luatex or xelatex
  \usepackage{unicode-math}
  \defaultfontfeatures{Ligatures=TeX,Scale=MatchLowercase}
\fi
% use upquote if available, for straight quotes in verbatim environments
\IfFileExists{upquote.sty}{\usepackage{upquote}}{}
% use microtype if available
\IfFileExists{microtype.sty}{%
\usepackage[]{microtype}
\UseMicrotypeSet[protrusion]{basicmath} % disable protrusion for tt fonts
}{}
\IfFileExists{parskip.sty}{%
\usepackage{parskip}
}{% else
\setlength{\parindent}{0pt}
\setlength{\parskip}{6pt plus 2pt minus 1pt}
}
\usepackage{hyperref}
\hypersetup{
            pdftitle={If psychological theory did not exist we would have to invent it},
            pdfauthor={Danielle J. Navarro},
            pdfkeywords={X},
            pdfborder={0 0 0},
            breaklinks=true}
\urlstyle{same}  % don't use monospace font for urls
\usepackage{graphicx,grffile}
\makeatletter
\def\maxwidth{\ifdim\Gin@nat@width>\linewidth\linewidth\else\Gin@nat@width\fi}
\def\maxheight{\ifdim\Gin@nat@height>\textheight\textheight\else\Gin@nat@height\fi}
\makeatother
% Scale images if necessary, so that they will not overflow the page
% margins by default, and it is still possible to overwrite the defaults
% using explicit options in \includegraphics[width, height, ...]{}
\setkeys{Gin}{width=\maxwidth,height=\maxheight,keepaspectratio}
\setlength{\emergencystretch}{3em}  % prevent overfull lines
\providecommand{\tightlist}{%
  \setlength{\itemsep}{0pt}\setlength{\parskip}{0pt}}
\setcounter{secnumdepth}{0}
% Redefines (sub)paragraphs to behave more like sections
\ifx\paragraph\undefined\else
\let\oldparagraph\paragraph
\renewcommand{\paragraph}[1]{\oldparagraph{#1}\mbox{}}
\fi
\ifx\subparagraph\undefined\else
\let\oldsubparagraph\subparagraph
\renewcommand{\subparagraph}[1]{\oldsubparagraph{#1}\mbox{}}
\fi

% set default figure placement to htbp
\makeatletter
\def\fps@figure{htbp}
\makeatother

\shorttitle{Psychological theory}
\affiliation{
\vspace{0.5cm}
\textsuperscript{1} School of Psychology, University of New South Wales}
\keywords{X\newline\indent Word count: X}
\usepackage{csquotes}
\usepackage{upgreek}
\captionsetup{font=singlespacing,justification=justified}

\usepackage{longtable}
\usepackage{lscape}
\usepackage{multirow}
\usepackage{tabularx}
\usepackage[flushleft]{threeparttable}
\usepackage{threeparttablex}

\newenvironment{lltable}{\begin{landscape}\begin{center}\begin{ThreePartTable}}{\end{ThreePartTable}\end{center}\end{landscape}}

\makeatletter
\newcommand\LastLTentrywidth{1em}
\newlength\longtablewidth
\setlength{\longtablewidth}{1in}
\newcommand{\getlongtablewidth}{\begingroup \ifcsname LT@\roman{LT@tables}\endcsname \global\longtablewidth=0pt \renewcommand{\LT@entry}[2]{\global\advance\longtablewidth by ##2\relax\gdef\LastLTentrywidth{##2}}\@nameuse{LT@\roman{LT@tables}} \fi \endgroup}

\title{If psychological theory did not exist we would have to invent it}
\author{Danielle J. Navarro\textsuperscript{1}}
\date{}

\authornote{This manuscript is based on conversations with Berna Dezever and many others. I want to specifically note Berna's contribution in this initial submission as she will likely be a coauthor on any published version. At the current point in development she has not had the opportunity to provide input and (as a way of assuming sole responsibilities for any errors in the current version) I have not listed her as a coauthor at this stage.

Correspondence concerning this article should be addressed to Danielle J. Navarro, School of Psychology, University of New South Wales, Kensington 2052, Sydney, Australia. E-mail: \href{mailto:d.navarro@unsw.edu.au}{\nolinkurl{d.navarro@unsw.edu.au}}}

\abstract{
Abstract is coming


}

\begin{document}
\maketitle

In 1987 Roger Shepard published a short paper in \emph{Science} with the ambitious title \enquote{Toward a Universal Law of Generalization for Psychological Science} (Shepard, 1987). Drawing on extensive work in the empirical literature on stimulus generalization across many species, he asserted the claim that the form of any stimulus generalization function shouold be approximately exponential in form, when measured with respect to an appropriately formulated stimulus representation. His paper begins with the following remark:

\begin{quote}
The tercentenary of the publication, in 1687, of Newton's \emph{Principia} prompts the question of whether psychological science has any hope of achieving a law that is comparable in generality (if not in predictive accuracy) to Newton's universal law of gravitation. Exploring the direction that currently seems most favorable for an affirmative answer, I outline empirical evidence an a theoretical rationale in support of a tentative candidate for a universal law of generalization
\end{quote}

Shepard's claim in the original paper was remarkable in scope. He drew on data from several terrestrial species (e.g., humans, pigeons, rats) and across many stimulus domains (e.g., visual, auditory), data that had hitherto been considered unrelated. To spot the connection between these data, Shepard used statistical insights from the similarity modelling literature. He noted that the apparent noninvariance of observed stimulus generalisation functions stemmed largely from the fact that response data were previously analysed with respect to the physical dissimilarities of the stimulus. When the same responses were replotted as a function of distance in a psychological space contructed by multidimensional scaling, he found that the form of the stimulus generalisation was remarkably regular in shape, as shown on the left side of Figure 1.

Taken by itself Shepard's empirical discovery would have been impressive. However, not content merely to discover the underlying invariance and thereby unify several distinct strands of research, Shepard went on to provide a theoretical explanation for \emph{why} we should expect to find this invariance. The theory was surprisingly simple: the learner presumes there exists some unknown \emph{consequential} region of the stimulus space across which roughly the same properties hold (e.g., things that look like apples will probably taste the same as one another). Encountering a single stimulus that entails a particular consequence, the learner's task is to infer the location, shape and size of the consequential region itself. Naturally this is an underconstrained problem, as there are an infinite number of possible regions that might correspond to the true consequential region. Nevertheless, Shepard showed that under a quite range of prior assumptions that the learner makes about the nature of consequential regions, the shape of the \emph{generalization} function across the stimulus space ends up approximately exponential, shown on the right side of Figure 1.



\begin{figure}[t]
\includegraphics[width=6.4in]{shepard} \caption{Two of the figures from Shepard's (1987) paper on stimulus generalisation. On the left, his Fig 1 shows the empirical evidence for an exponential law and on the right his Fig 3 outlines his theoretical claim. {[}Note: higher resolution versions will be provided in the final version and permission will be obtained to reproduce the originals{]}}\label{fig:unnamed-chunk-1}
\end{figure}

Since its publication in 1987, Shepard's paper has been cited approximately 2500 times according to Google Scholar, no small feat for a theoretical paper that presented no new empirical data and whose content is almost entirely devoted to the derivation of a formal relation between one unobservable quantity (distance between two items psychological space) and another (the probability two items fall in the same region).

\hypertarget{references}{%
\section{References}\label{references}}

\begingroup
\setlength{\parindent}{-0.5in}
\setlength{\leftskip}{0.5in}

\hypertarget{refs}{}
\leavevmode\hypertarget{ref-shepard1987toward}{}%
Shepard, R. N. (1987). Toward a universal law of generalization for psychological science. \emph{Science}, \emph{237}(4820), 1317--1323.

\leavevmode\hypertarget{ref-shepard1987toward}{}%
Shepard, R. N. (1987). Toward a universal law of generalization for psychological science. \emph{Science}, \emph{237}(4820), 1317--1323.

\endgroup

\end{document}
